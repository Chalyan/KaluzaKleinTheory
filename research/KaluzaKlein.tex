%% LyX 2.3.7 created this file.  For more info, see http://www.lyx.org/.
%% Do not edit unless you really know what you are doing.
\documentclass[12pt,american]{article}
\usepackage{amsmath}
\usepackage{amsthm}
\usepackage{fontspec}
\setmainfont[Mapping=tex-text]{Times New Roman}
\setsansfont[Mapping=tex-text]{Times New Roman}
\setmonofont{Times New Roman}
\usepackage[a4paper]{geometry}
\geometry{verbose}

\makeatletter
\@ifundefined{date}{}{\date{}}
%%%%%%%%%%%%%%%%%%%%%%%%%%%%%% User specified LaTeX commands.
\usepackage{amsthm}
\usepackage{fontspec}
\usepackage{geometry}
 \geometry{
 right=12mm,
 left=12mm,
 top=17.5mm,
 bottom=22.5mm
 }
\renewcommand{\baselinestretch}{1.5} 
\sloppy

\makeatother

\usepackage{polyglossia}
\setdefaultlanguage[variant=american]{english}
\begin{document}
\title{ԿԱԼՈՒԶԱ-ՔԼԱՅՆԻ ԳՐԱՎԻՏԱՑԻԱ}
\author{Հետազոտական աշխատանք\\
 Չալյան Գոռ Վարդանի\\
 Ակադեմիկոս Գ․ Սահակյանի անվան Տեսական Ֆիզիկայի ամբիոն\\
 Մագիստրատուրա 1 կուրս}

\maketitle
\thispagestyle{empty}

\newpage{}

\section{ՆԵՐԱԾՈՒԹՅՈՒՆ}

$\;$

Կալուզայի ձեռքբերումը 5-աչափ տարածությունում ձևակերպված հարաբարականության
ընդհանուր տեսության՝ Էյնշտեյնի 4-աչափ գրավիտացիայի, ինչպես նաև Մաքսվելի
էլեկտրամագնիսականության տեսություններն իր մեջ պարունակել ցույց տալն
էր։ Սակայն նա մտցրեց արհեստական թվացող մի պայման, որը կոչվում է այսպես
կոչված \textbf{գլանային պայման}՝ դրված նոր մտցված 5-րդ կոորդինատի
վրա։ Վերջինս նշանակում է, որ ուղղակի ազդեցություն չունի ֆիզիկայի օրենքների
վրա։ Քլայնի ներդրումն այս պայմանի քիչ արհեստական դարձնելն էր, առաջարկելով
կոմպակտիֆիկացնել նոր՝ 5-րդ չափողականությունը։ Այս գաղափարը մեծ ցնծությամբ
ընդունվեց միացյալ տեսություն ունենալ ցանկացողների շրջանում, և երբ
հերթն արդեն հասել էր ուժեղ և թույլ փոխազդեցություններն էլ միավորելուն,
ենթադրվեց, որ ըստ Կալուզայի տրամաբանության ավելացված ավել չափողականությունները
ևս պետք է լինեն կոմպակտ։ Այս մտածելակերպը հիմք դարձավ 1980-ականներին
11 չափանի սուպեր-գրավիտացիայի տեսության, իսկ հետագայում՝ «Ամեն ինչի
տեսության»։

Կալուզան միավորեց ոչ միայն էլեկտրամագնիսականությունն ու գրավիտացիան,
այլ նաև նյութը և տարածության երկրաչափությունը՝ 4-աչափ տարածությունում
գտնվող ֆոտոնը դիտարկելով որպես դատարկություն՝ 5-աչափ տարածությունում։
Ժամանակակից Կալուզա-Քլայնի տեսությունները պահանջում են բարձր չափողականությամբ
նյութի դաշտերի առկայություն՝ հաջող կոմպակտիֆիկացիա համար։

Հարց է առաջանում, արդյո՞ք սա պարտադիր պայման է, թե՞ ոչ։ Պատասխանն
«այո» է, եթե դնում ենք ավել չափողականությունների \textbf{իրական},
\textbf{կոմպակտ} և \textbf{տարածանման} լինելու պայմանները։ Սակայն
գոյություն ունեն այլ տեսություններ, որոնցում այս պայմանները չեն պահանջվում,
օրինակ \textbf{պռոյեկտվող տեսությունները,} որոնցում չի պահանջվում
ավել չափողականությունների իրական լինելը, կամ \textbf{ոչ-կոմպակտիֆիկացված
տեսությունները}, որոնցում չի պահանջվում ավել չափողականությունների
տարածանման, կամ կոմպակտ լինելը։

\section{ԲԱՐՁՐ ՉԱՓՈՂԱԿԱՆՈՒԹՅՈՒՆՆԵՐ}

$\;$

Մեր ամենօրյա աշխարհը 3-աչափ է։ Բայց հարց է առաջանում, թե ինչու՞ է
այդպես։ Այս հարցին փորձել է պատասխանել դեռևս Կեպլերը, ով այս հանգամանքի
պատճառը փիլիսոփայորեն հնարավոր էր համարում կապել Սուրբ երրորդության
հետ։ Ավելի հին պատճառաբանումներից են մոլորակային ուղեծրերի կայունությունը,
ատոմում հիմնական վիճակները, բնության մեջ առկա որոշ ֆունդամենտալ հաստատուններ,
ինֆորմացիայի փոխանցման համար ալիքների տարածումը և այլն։ Թվարկվածները
հանգում են համաձայնության, որ տարածությունը կազմված է 3 տարածական
մակրոսկոպական կոորդինատներով՝ $x^{1},x^{2},x^{3}$։

Բարձր չափողականություններով տեսություններում ֆիզիկայի օրենքներն ընդունում
են պարզ ձևակերպումներ, որոնք 3-աչափ դիտարկելիս հանգեցնում էին բարդ
արտահայտությունների։ Սակայն ինչպե՞ս պետք է համատեղել այս գաղափարը
3-աչափ տարածության հետ։ Եվ եթե իսկապես կան ավել չափողականություններ,
ինչպե՞ս է ստացվում, որ ֆիզիկան դրանցից կախված չէ։

Պետք է չմոռանանք, որ նոր մտցված ավել կոորդինատները պարտադիր չէ, որ
ունենան հեռավորության չափողականություն, կամ լինեն տարածանման՝ հաշվի
առնելով մետրիկայում այդ անդամի նշանը։ Ավելին, Մինկովսկին 1909 թվականին
ներմուծեց հենց այս երկու հանգամանքին հակասող օրինակ, որում Մաքսվելի
միացյալ էլեկտրամագնիսականության ու Էյնշտեյնի հարաբերականության հատուկ
տեսությունները հնարավոր եղան պատկերացնել երկրաչափորեն, եթե ժամանակը՝
տարածական կոորդինատների հետ միասին համարվեն 4-աչափ տարածություն, $x^{0}=ict$-ի
ներմուծմամբ։ Երկար ժամանակ ֆիզիկայի՝ 3 տարածական կոորդինատներից կախված
լինելու պատճառը $c$ պարամետրի մեծ արժեք ունենալն էր, որի պատճառով
տարածական և ժամանակային կոորդինատների «խառնվելը» հանդես էր գալիս միայն
մեծ արագությունների դեպքում։

\section{ԿԱԼՈՒԶԱ-ՔԼԱՅՆԻ ՏԵՍՈՒԹՅՈՒՆԸ}

$\;$

Ոգեշնչված Մինկովսկու՝ տարածության 4-աչափ նկարագրությունից, իրարից
անկախ Նորդստրոմը՝ 1914 թ․-ին և Կալուզան՝ 1921 թ․-ին առաջիններից էին,
ովքեր փորձեցին միավորել գրավիտացիան և էլեկտրամագնիսականությունը՝ 5-աչափ
տարածության պատկերացմամբ, ավելացնելով $x^{4}$ տարածական անդամը։ Երկուսի
մոտ էլ առաջացավ նույն հարցը․ ինչու՞ այդ 4-րդ տարածական կոորդինատը
չի նշմարվում բնության մեջ։

Դեռևս Մինկովսկու ժամանակներից երևույթների՝ Լորենց ինվարիանտ լինելը
մեկնաբանվում էր 4-աչափ կոորդինատների ինվարիանտությամբ։ 5-աչափ տարածությունում
նույնը չէր դիտվում, այդ իսկ պատճառով Նորդստրոմն ու Կալուզան զերծ էին
մնում այդ հարցից և պարզապես դրեցին պահանջ, որ բոլոր ածանցյալներն ըստ
$x^{4}$-ի վերանան։ Այլ ձևակերպմամբ՝ ֆիզիկան պետք է դրսևորվեր 5-աչափ
տարածության 4-աչափ հիպերմակաերևույթի վրա(Կալուզայի գլանային պայման)։

Այս ենթադրությամբ, նրանցից յուրաքանչյուրին հաջողվեց ստանալ էլեկտրամագնիսականության
և գրավիտացիայի հավասարումները՝ 5-աչափ տեսությունից։ Նորդստրոմը ենթադրեց
սկալյար գրավիտացիոն պոտենցյալ(աշխատանքները սկսվել էին մինչև հարաբերականության
ընդհանուր տեսությունը), իսկ Կալուզան օգտագործեց Էյնշտեյնի թենզոր պոտենցյալը։
Ավելին, Կալուզան ցույց տվեց, որ հարաբերականության ընդհանուր տեսությունը,
երբ դիտարկվում է որպես վակուումում 5-աչափ տեսություն($^{5}G_{AB}=0,\;A,B=0,1,2,3,4$),
պարունակում է 4-աչափ հարաբերականության ընդհանուր տեսություն՝ էլեկտրամագնիսական
դաշտի առկայությամբ, Մաքսվելի հավասարումների հետ միասին($^{4}G_{\alpha\beta}=^{4}T_{\alpha\beta}^{EM},\;\alpha,\beta=0,1,2,3$)։

Առաջարկվեցին Կալուզայի 5-աչափ տեսության սխեմայի տարբեր մոդիֆիկացիաներ։
Օրինակ ավել չափողականությունը կոմպակտիֆիկացնելն առաջարկեցին Էյնշտեյնը,
Ջորդանը, Բերգմանը և այլոք։ Սխեման չընդլայնվեց ավելի բարձր չափողականությամբ
տեսությունների վրա, քանի դեռ չմշակվեցին ուժեղ և թույլ փոխազդեցությունների
տեսությունները։ Հարցն այն էր, թե արդյոք այս նոր ուժերը կարո՞ղ էին
միավորվել էլեկտրամագնիսականության և գրավիտացիան նույն մեթոդով, թե
ոչ։

Պատասխանը թաքնված էր տրամաչափային ինվարիանտության հիմքում։ Օրինակ
էլեկտրադինամիկան կստացվի կստացվի կիրառելով $U(1)$ տրամաչափային ինվարիանտություն
ազատ մասնիկի Լագրանժյանի վրա։ Այս հանգամանքն այդքան էլ զարմանալի չէ,
քանի որ $U(1)$ խումբն «ավելացել» է Էյնշտեյնի հավասարումներում՝ 5-րդ
չափողականության կոորդինատական ձևափոխությունների նկատմամբ ինվարիանտության
անվան տակ։ Այլ կերպ ասած, տրամաչափային ինվարիանտությունը բացատրվել
է որպես տարածաժամանակի երկրաչափական սիմետրիա։ Էլեկտրամագնիսական դաշտն
այնուհետև հանդես եկավ որպես վեկտորական տրամաչափային դաշտ՝ 4-աչափ տարածությունում։
Բարդ էր սա ավելի խճճված սիմետրիայով խմբերի համար ընդհանրացնելը։ 1963
թ․-ին առաջին անգամ առաջարկվեց ներառել ոչ-Աբելյան $SU(2)$ խումբը $(4+d)$
չափանի Կալուզա-Քլայնի տեսությունում։ Ամենաքիչը 3 նոր չափողականություն
անհրաժեշտ էր։ Դրված խնդիրն ամբողջովին լուծվեց 1975 թ․-ին։ 

\section{ԲԱՐՁՐ ՉԱՓԱՆԻ ՄԻԱՎՈՐՈՒՄՆԵՐԻ ՄՈՏԵՑՈՒՄԸ}

$\;$

Նշենք 3 հիմնական սկզբունքները, որոնք մինչ այս դիտարկված մոդելներում
հաշվի են առնվել․
\begin{enumerate}
\item Էյնշտեյնի պատկերացումը՝ բնությունը մաքուր երկրաչափություն դիտարկելու։
Էլեկտրամագնիսական, գրավիտացիոն և Յանգ-Միլսի դաշտերն ամբողջովին պարունակվում
են բարձր չափանի Էյնշտեյնի $^{(4+d)}G_{AB}$ թենզորում, և րացուցիչ
$^{(4+d)}T_{AB}$ էներգիա-իմպուլսի թենզոր չի պահանջվում։
\item Լրացուցիչ մաթեմատիկական անդամներ ավելացված չեն Էյնշտեյնի հարաբերականության
ընդհանուր տեսության թենզորական արտահայտություններին, և միակ փոփոխությունն
այն է, որ թենզորական հավասարումներում ինդեքսները փոփոխվում են 0-ից
մինչև (3+d), որտեղ d-ն մտցված ավել չափողականությունների քանակն է։
\item Մտցված կոորդինատները ենթարկվում են գլանային պայմանին և ներդրում չունեն
երևույթների ֆիզիկայի վրա։ Չկա ոչ մի մեխանիզմ, որը կբացատրի միայն առաձին
4 կոորդինատների ներդրումը։
\end{enumerate}
%
Առաջին 2 սկզբունքներն ընդունելի եղան էլեգանտության և պարզության տեսանկյունից։
Սակայն դրան զուգագեռ 3-րդն այդքան էլ մեր այսօրվա աչքին սովոր չէ։

Դեռևս Կալուզայի ժամանակներից այս պայմանը մեղմացնելու համար, բարձր
չափանի միավորման տեսությունները զարգացան 3, համեմատաբար անկախ ուղղություններով,
որոնցից յուրաքանչյուրը ցավոք կորցնում էր վերոնշյալ 3 պայմաններից որևէ
մեկը։

Առաջին եղանակը ավել չափողականությունների կոմպակտ լինելն է, որի արդյունքում
էլ տեսանելի չեն լինում և չեն գտնվում էներգիայի՝ փորձնականորեն հասանելի
տիրույթներում։ Այս դեպքում խնդիրն առաջանում է, երբ փորձեր են կատարում
միավորել ոչ միայն գրավիտացիան և էլեկտրամագնիսականությունն, այլ նաև
այլ փոխազդեցություններ, և այդ դեպքում խախտվում է 1-ին, Էյնշտեյնի՝
ֆիզիկան մաքուր երկրաչափական նկարագրությամբ պատկերացնելու սկզբունքը։

Երկրորդ եղանակի դեպքում խախտվում է 2-րդ սկզբունքը և լրացուցիչ անդամները
դիտարկվում են որպես ավելի բարդ տեսության անդամներ։ Սա կարող է կատարվել
Էյնշտեյնի հարաբերականության ընդհանուր տեսության երկրաչափությունը փոխարինելով
պրոյեկտվող երկրաչափությամբ։ Լրացուցիչ չափողականությունները դառնում
են զուտ տեսանելի օգնություն, որոնք ցավոք կարող է և բացատրելի չդարձնեն
բնության հիմքում ընկած մաթեմատիկան։

Երրորդ եղանակի դեպքում 3-րդ՝ գլանային պայմանի անհստակեցվածություն
է առաջանում։ Այս դեպքում թույլատվում է, որ ֆիզիկան կախված լինի նոր
կոորդինատներից, սակայն նրանց ազդեցությունները չեն զգացվում փորձնականորեն
հասանելի երևույթների դիտարկման ժամանակ, ինչպես օրինակ Մինկովսկու դեպքում
ոչ-ռելյատիվիստական արագությունների ժամանակ 4-րդ կոորդինատը զգացնել
չէր տալիս։ Երբ հաշվի է առնվում ֆիզիկայի՝ լրացուցիչ չափողականություններից
կախվածությունը, 5-աչափ Էյնշտեյնի $^{5}R_{AB}=0$ հավասարումներն իրենց
մեջ պարունակում են 4-աչափ $^{4}G_{\alpha\beta}=^{4}T_{\alpha\beta}$
հավասարումները, որտեղ $^{4}T_{\alpha\beta}$-ն ընդհանուր էներգիա-իմպուլսի
թենզորն է, ոչ թե միայն էլեկտրամագնիսականության $^{4}T_{\alpha\beta}^{EM}$
թենզորը։

\section{ԿՈՄՊԱԿՏԻՖԻԿԱՑՄԱՆ ՄՈՏԵՑՈՒՄԸ}

$\;$

Քլայնը 1926 թ․-ին ցույց տվեց, որ Կալուզայի գլանային պայմանը բնական
կերպով կառաջանար, եթե 5-րդ կոորդինատն ունենար շրջանային տոպոլոգիա,
որի դեպքում ֆիզիկան դրանից կախված կլիներ միայն պարբերական կերպով և
հնարավոր կլիներ ենթարկել Ֆուիրե ձևափոխության, և այնքան փոքր, որ բացի
հիմնական վիճակին համապատասխան մոդից բացի մնացած այլ մոդերի էներգիան
լիներ այնքան մեծ, որ դիտելի չլիներ՝ բացառությամբ վաղ տիեզերքի։ Ֆիզիկան
էֆեկտիվորեն կախված չի լինի Կալուզայի 5-րդ չափողականությունից, ինչն
էլ հենց պահանջվում էր։
\begin{thebibliography}{1}
\bibitem{key-1}

\end{thebibliography}

\end{document}
