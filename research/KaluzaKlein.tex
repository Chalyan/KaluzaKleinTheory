%% LyX 2.3.7 created this file.  For more info, see http://www.lyx.org/.
%% Do not edit unless you really know what you are doing.
\documentclass[12pt,american]{article}
\usepackage{amsmath}
\usepackage{amsthm}
\usepackage{fontspec}
\setmainfont[Mapping=tex-text]{Times New Roman}
\setsansfont[Mapping=tex-text]{Times New Roman}
\setmonofont{Times New Roman}
\usepackage[a4paper]{geometry}
\geometry{verbose}

\makeatletter
\@ifundefined{date}{}{\date{}}
%%%%%%%%%%%%%%%%%%%%%%%%%%%%%% User specified LaTeX commands.
\usepackage{amsmath}
\usepackage{latexsym}
\usepackage{geometry}
 \geometry{
 right=15mm,
 left=15mm,
 top=20mm,
 bottom=25mm
 }
\renewcommand{\baselinestretch}{1.5} 
\sloppy

\makeatother

\usepackage{polyglossia}
\setdefaultlanguage[variant=american]{english}
\begin{document}
\title{ԿԱԼՈՒԶԱ-ՔԼԱՅՆԻ ԳՐԱՎԻՏԱՑԻԱ}
\author{Հետազոտական աշխատանք\\
 Չալյան Գոռ Վարդանի\\
 Ակադեմիկոս Գ․ Սահակյանի անվան Տեսական Ֆիզիկայի ամբիոն\\
 Մագիստրատուրա 1 կուրս}

\maketitle
\renewcommand{\contentsname}{Բովանդակություն}
\renewcommand{\refname}{Գրականություն}
\thispagestyle{empty}

\newpage{}

\tableofcontents{}

\newpage

\section{ՆԵՐԱԾՈՒԹՅՈՒՆ}

$\;$

Կալուզայի ձեռքբերումը 5-աչափ տարածությունում ձևակերպված հարաբարականության
ընդհանուր տեսության՝ Էյնշտեյնի 4-աչափ գրավիտացիայի, ինչպես նաև Մաքսվելի
էլեկտրամագնիսականության տեսություններն իր մեջ պարունակել ցույց տալն
էր։ Սակայն նա մտցրեց արհեստական թվացող մի պայման, որը կոչվում է այսպես
կոչված \textbf{գլանային պայման}՝ դրված նոր մտցված 5-րդ կոորդինատի
վրա։ Վերջինս նշանակում է, որ ուղղակի ազդեցություն չունի ֆիզիկայի օրենքների
վրա։ Քլայնի ներդրումն այս պայմանի քիչ արհեստական դարձնելն էր, առաջարկելով
կոմպակտիֆիկացնել նոր՝ 5-րդ չափողականությունը։ Այս գաղափարը մեծ ցնծությամբ
ընդունվեց միացյալ տեսություն ունենալ ցանկացողների շրջանում, և երբ
հերթն արդեն հասել էր ուժեղ և թույլ փոխազդեցություններն էլ միավորելուն,
ենթադրվեց, որ ըստ Կալուզայի տրամաբանության ավելացված ավել չափողականությունները
ևս պետք է լինեն կոմպակտ։ Այս մտածելակերպը հիմք դարձավ 1980-ականներին
11 չափանի սուպեր-գրավիտացիայի տեսության, իսկ հետագայում՝ «Ամեն ինչի
տեսության»։

Կալուզան միավորեց ոչ միայն էլեկտրամագնիսականությունն ու գրավիտացիան,
այլ նաև նյութը և տարածության երկրաչափությունը՝ 4-աչափ տարածությունում
գտնվող ֆոտոնը դիտարկելով որպես դատարկություն՝ 5-աչափ տարածությունում։
Ժամանակակից Կալուզա-Քլայնի տեսությունները պահանջում են բարձր չափողականությամբ
նյութի դաշտերի առկայություն՝ հաջող կոմպակտիֆիկացիա համար։

Հարց է առաջանում, արդյո՞ք սա պարտադիր պայման է, թե՞ ոչ։ Պատասխանն
«այո» է, եթե դնում ենք ավել չափողականությունների \textbf{իրական},
\textbf{կոմպակտ} և \textbf{տարածանման} լինելու պայմանները։ Սակայն
գոյություն ունեն այլ տեսություններ, որոնցում այս պայմանները չեն պահանջվում,
օրինակ \textbf{պռոյեկտվող տեսությունները,} որոնցում չի պահանջվում
ավել չափողականությունների իրական լինելը, կամ \textbf{ոչ-կոմպակտիֆիկացված
տեսությունները}, որոնցում չի պահանջվում ավել չափողականությունների
տարածանման, կամ կոմպակտ լինելը։

Այս հետազոտական աշխատանքում համառոտ ձևակերպված և թարգմանված է \cite{KKG}
գրականության որոշ բաժիններ, որոնցից են Կալուզա-Քլայնի տեսության պատմական
ակնարկը՝ առաջացման պատճառները, տեսության մշակման ժամանակ առաջ եկած
որոշ խնդիրների մոտեցումներ և այլն։ \cite{KKG}-ից բացի ներառված չեն
այլ ցիտումներ, քանի որ դրանք արդեն առկա են հոդվածի բնօրինակում և լրացուցիչ
աղբյուրներից տեղեկություններ ավելացված չեն։ 

\section{ԲԱՐՁՐ ՉԱՓՈՂԱԿԱՆՈՒԹՅՈՒՆՆԵՐ}

$\;$

Մեր ամենօրյա աշխարհը 3-աչափ է։ Բայց հարց է առաջանում, թե ինչու՞ է
այդպես։ Այս հարցին փորձել է պատասխանել դեռևս Կեպլերը, ով այս հանգամանքի
պատճառը փիլիսոփայորեն հնարավոր էր համարում կապել Սուրբ երրորդության
հետ։ Ավելի հին պատճառաբանումներից են մոլորակային ուղեծրերի կայունությունը,
ատոմում հիմնական վիճակները, բնության մեջ առկա որոշ ֆունդամենտալ հաստատուններ,
ինֆորմացիայի փոխանցման համար ալիքների տարածումը և այլն։ Թվարկվածները
հանգում են համաձայնության, որ տարածությունը կազմված է 3 տարածական
մակրոսկոպական կոորդինատներով՝ $x^{1},x^{2},x^{3}$։

Բարձր չափողականություններով տեսություններում ֆիզիկայի օրենքներն ընդունում
են պարզ ձևակերպումներ, որոնք 3-աչափ դիտարկելիս հանգեցնում էին բարդ
արտահայտությունների։ Սակայն ինչպե՞ս պետք է համատեղել այս գաղափարը
3-աչափ տարածության հետ։ Եվ եթե իսկապես կան ավել չափողականություններ,
ինչպե՞ս է ստացվում, որ ֆիզիկան դրանցից կախված չէ։

Պետք է չմոռանանք, որ նոր մտցված ավել կոորդինատները պարտադիր չէ, որ
ունենան հեռավորության չափողականություն, կամ լինեն տարածանման՝ հաշվի
առնելով մետրիկայում այդ անդամի նշանը։ Ավելին, Մինկովսկին 1909 թվականին
ներմուծեց հենց այս երկու հանգամանքին հակասող օրինակ, որում Մաքսվելի
միացյալ էլեկտրամագնիսականության ու Էյնշտեյնի հարաբերականության հատուկ
տեսությունները հնարավոր եղան պատկերացնել երկրաչափորեն, եթե ժամանակը՝
տարածական կոորդինատների հետ միասին համարվեն 4-աչափ տարածություն, $x^{0}=ict$-ի
ներմուծմամբ։ Երկար ժամանակ ֆիզիկայի՝ 3 տարածական կոորդինատներից կախված
լինելու պատճառը $c$ պարամետրի մեծ արժեք ունենալն էր, որի պատճառով
տարածական և ժամանակային կոորդինատների «խառնվելը» հանդես էր գալիս միայն
մեծ արագությունների դեպքում։

\section{ԿԱԼՈՒԶԱ-ՔԼԱՅՆԻ ՏԵՍՈՒԹՅՈՒՆԸ}

$\;$

Ոգեշնչված Մինկովսկու՝ տարածության 4-աչափ նկարագրությունից, իրարից
անկախ Նորդստրոմը՝ 1914 թ․-ին և Կալուզան՝ 1921 թ․-ին առաջիններից էին,
ովքեր փորձեցին միավորել գրավիտացիան և էլեկտրամագնիսականությունը՝ 5-աչափ
տարածության պատկերացմամբ, ավելացնելով $x^{4}$ տարածական անդամը։ Երկուսի
մոտ էլ առաջացավ նույն հարցը․ ինչու՞ այդ 4-րդ տարածական կոորդինատը
չի նշմարվում բնության մեջ։

Դեռևս Մինկովսկու ժամանակներից երևույթների՝ Լորենց ինվարիանտ լինելը
մեկնաբանվում էր 4-աչափ կոորդինատների ինվարիանտությամբ։ 5-աչափ տարածությունում
նույնը չէր դիտվում, այդ իսկ պատճառով Նորդստրոմն ու Կալուզան զերծ էին
մնում այդ հարցից և պարզապես դրեցին պահանջ, որ բոլոր ածանցյալներն ըստ
$x^{4}$-ի վերանան։ Այլ ձևակերպմամբ՝ ֆիզիկան պետք է դրսևորվեր 5-աչափ
տարածության 4-աչափ հիպերմակաերևույթի վրա(Կալուզայի գլանային պայման)։

Այս ենթադրությամբ, նրանցից յուրաքանչյուրին հաջողվեց ստանալ էլեկտրամագնիսականության
և գրավիտացիայի հավասարումները՝ 5-աչափ տեսությունից։ Նորդստրոմը ենթադրեց
սկալյար գրավիտացիոն պոտենցյալ(աշխատանքները սկսվել էին մինչև հարաբերականության
ընդհանուր տեսությունը), իսկ Կալուզան օգտագործեց Էյնշտեյնի թենզոր պոտենցյալը։
Ավելին, Կալուզան ցույց տվեց, որ հարաբերականության ընդհանուր տեսությունը,
երբ դիտարկվում է որպես վակուումում 5-աչափ տեսություն($^{5}G_{AB}=0,\;A,B=0,1,2,3,4$),
պարունակում է 4-աչափ հարաբերականության ընդհանուր տեսություն՝ էլեկտրամագնիսական
դաշտի առկայությամբ, Մաքսվելի հավասարումների հետ միասին($^{4}G_{\alpha\beta}=^{4}T_{\alpha\beta}^{EM},\;\alpha,\beta=0,1,2,3$)։

Առաջարկվեցին Կալուզայի 5-աչափ տեսության սխեմայի տարբեր մոդիֆիկացիաներ։
Օրինակ ավել չափողականությունը կոմպակտիֆիկացնելն առաջարկեցին Էյնշտեյնը,
Ջորդանը, Բերգմանը և այլոք։ Սխեման չընդլայնվեց ավելի բարձր չափողականությամբ
տեսությունների վրա, քանի դեռ չմշակվեցին ուժեղ և թույլ փոխազդեցությունների
տեսությունները։ Հարցն այն էր, թե արդյոք այս նոր ուժերը կարո՞ղ էին
միավորվել էլեկտրամագնիսականության և գրավիտացիան նույն մեթոդով, թե
ոչ։

Պատասխանը թաքնված էր տրամաչափային ինվարիանտության հիմքում։ Օրինակ
էլեկտրադինամիկան կստացվի կստացվի կիրառելով $U(1)$ տրամաչափային ինվարիանտություն
ազատ մասնիկի Լագրանժյանի վրա։ Այս հանգամանքն այդքան էլ զարմանալի չէ,
քանի որ $U(1)$ խումբն «ավելացել» է Էյնշտեյնի հավասարումներում՝ 5-րդ
չափողականության կոորդինատական ձևափոխությունների նկատմամբ ինվարիանտության
անվան տակ։ Այլ կերպ ասած, տրամաչափային ինվարիանտությունը բացատրվել
է որպես տարածաժամանակի երկրաչափական սիմետրիա։ Էլեկտրամագնիսական դաշտն
այնուհետև հանդես եկավ որպես վեկտորական տրամաչափային դաշտ՝ 4-աչափ տարածությունում։
Բարդ էր սա ավելի խճճված սիմետրիայով խմբերի համար ընդհանրացնելը։ 1963
թ․-ին առաջին անգամ առաջարկվեց ներառել ոչ-Աբելյան $SU(2)$ խումբը $(4+d)$
չափանի Կալուզա-Քլայնի տեսությունում։ Ամենաքիչը 3 նոր չափողականություն
անհրաժեշտ էր։ Դրված խնդիրն ամբողջովին լուծվեց 1975 թ․-ին։ 

\section{ԲԱՐՁՐ ՉԱՓԱՆԻ ՄԻԱՎՈՐՈՒՄՆԵՐԻ ՄՈՏԵՑՈՒՄԸ}

$\;$

Նշենք 3 հիմնական սկզբունքները, որոնք մինչ այս դիտարկված մոդելներում
հաշվի են առնվել․
\begin{enumerate}
\item Էյնշտեյնի պատկերացումը՝ բնությունը մաքուր երկրաչափություն դիտարկելու։
Էլեկտրամագնիսական, գրավիտացիոն և Յանգ-Միլսի դաշտերն ամբողջովին պարունակվում
են բարձր չափանի Էյնշտեյնի $^{(4+d)}G_{AB}$ թենզորում, և րացուցիչ
$^{(4+d)}T_{AB}$ էներգիա-իմպուլսի թենզոր չի պահանջվում,
\item լրացուցիչ մաթեմատիկական անդամներ ավելացված չեն Էյնշտեյնի հարաբերականության
ընդհանուր տեսության թենզորական արտահայտություններին, և միակ փոփոխությունն
այն է, որ թենզորական հավասարումներում ինդեքսները փոփոխվում են 0-ից
մինչև (3+d), որտեղ d-ն մտցված ավել չափողականությունների քանակն է։
\item մտցված կոորդինատները ենթարկվում են գլանային պայմանին և ներդրում չունեն
երևույթների ֆիզիկայի վրա։ Չկա ոչ մի մեխանիզմ, որը կբացատրի միայն առաձին
4 կոորդինատների ներդրումը։
\end{enumerate}
%
Առաջին 2 սկզբունքներն ընդունելի եղան էլեգանտության և պարզության տեսանկյունից։
Սակայն դրան զուգագեռ 3-րդն այդքան էլ մեր այսօրվա աչքին սովոր չէ։

Դեռևս Կալուզայի ժամանակներից այս պայմանը մեղմացնելու համար, բարձր
չափանի միավորման տեսությունները զարգացան 3, համեմատաբար անկախ ուղղություններով,
որոնցից յուրաքանչյուրը ցավոք կորցնում էր վերոնշյալ 3 պայմաններից որևէ
մեկը։

Առաջին եղանակը ավել չափողականությունների կոմպակտ լինելն է, որի արդյունքում
էլ տեսանելի չեն լինում և չեն գտնվում էներգիայի՝ փորձնականորեն հասանելի
տիրույթներում։ Այս դեպքում խնդիրն առաջանում է, երբ փորձեր են կատարում
միավորել ոչ միայն գրավիտացիան և էլեկտրամագնիսականությունն, այլ նաև
այլ փոխազդեցություններ, և այդ դեպքում խախտվում է 1-ին, Էյնշտեյնի՝
ֆիզիկան մաքուր երկրաչափական նկարագրությամբ պատկերացնելու սկզբունքը։

Երկրորդ եղանակի դեպքում խախտվում է 2-րդ սկզբունքը և լրացուցիչ անդամները
դիտարկվում են որպես ավելի բարդ տեսության անդամներ։ Սա կարող է կատարվել
Էյնշտեյնի հարաբերականության ընդհանուր տեսության երկրաչափությունը փոխարինելով
պրոյեկտվող երկրաչափությամբ։ Լրացուցիչ չափողականությունները դառնում
են զուտ տեսանելի օգնություն, որոնք ցավոք կարող է և բացատրելի չդարձնեն
բնության հիմքում ընկած մաթեմատիկան։

Երրորդ եղանակի դեպքում 3-րդ՝ գլանային պայմանի անհստակեցվածություն
է առաջանում։ Այս դեպքում թույլատվում է, որ ֆիզիկան կախված լինի նոր
կոորդինատներից, սակայն նրանց ազդեցությունները չեն զգացվում փորձնականորեն
հասանելի երևույթների դիտարկման ժամանակ, ինչպես օրինակ Մինկովսկու դեպքում
ոչ-ռելյատիվիստական արագությունների ժամանակ 4-րդ կոորդինատը զգացնել
չէր տալիս։ Երբ հաշվի է առնվում ֆիզիկայի՝ լրացուցիչ չափողականություններից
կախվածությունը, 5-աչափ Էյնշտեյնի $^{5}R_{AB}=0$ հավասարումներն իրենց
մեջ պարունակում են 4-աչափ $^{4}G_{\alpha\beta}=^{4}T_{\alpha\beta}$
հավասարումները, որտեղ $^{4}T_{\alpha\beta}$-ն ընդհանուր էներգիա-իմպուլսի
թենզորն է, ոչ թե միայն էլեկտրամագնիսականության $^{4}T_{\alpha\beta}^{EM}$
թենզորը։

\section{ԿՈՄՊԱԿՏԻՖԻԿԱՑՄԱՆ ՄՈՏԵՑՈՒՄԸ և ՄԵԽԱՆԻԶՄՆԵՐԸ}

$\;$

Քլայնը 1926 թ․-ին ցույց տվեց, որ Կալուզայի գլանային պայմանը բնական
կերպով կառաջանար, եթե 5-րդ կոորդինատն ունենար շրջանային տոպոլոգիա,
որի դեպքում ֆիզիկան դրանից կախված կլիներ միայն պարբերական կերպով և
հնարավոր կլիներ ենթարկել Ֆուիրե ձևափոխության, և այնքան փոքր, որ բացի
հիմնական վիճակին համապատասխան մոդից բացի մնացած այլ մոդերի էներգիան
լիներ այնքան մեծ, որ դիտելի չլիներ՝ բացառությամբ վաղ տիեզերքի։ Ֆիզիկան
էֆեկտիվորեն կախված չի լինի Կալուզայի 5-րդ չափողականությունից, ինչն
էլ հենց պահանջվում էր։ Ավելին, թվում էր էլեկտրամագնիսական դաշտի Ֆուրիե
ձևափոխությունը կբացատրեր լիցքի քվանտացումը։

Հետևյալ սխեման կատարյալ չէր, դեռևս անհրաժեշ էր բացատրել ավել չափողականությունների
այդքան տարբերվելը՝ տոպոլոգիայով և մասշտաբով, մյուս տարածաժամանակայիններից։
Նրանց չափերը պետք է լինեին բավական փոքր՝ ատտոմետրերից քիչ($1\text{ամ}=10^{-18}\text{մ}$)։
Կար նաև տեսությունում նոր առաջացած սկալյար դաշտը բացատրելու խնդիր,
սակայն սկալյար դաշտերն այդքան էլ մեծ խնդիր չեն ներկայացնում իրենցից։

Քլայնի՝ ավել չափողականությունների կոմպակտիֆիկացման ռազմավարությունը
ազդեցիկություն ձեռք բերեց միավորյալ տեսություններում։

Կոմպակտիֆիկացման ժամանակ խնդիր է առաջանում այն կիրառել անխտիր բոլոր
չափողականությունների վրա, որոնք կցանկանանք․ մակրոսկոպական 4-աչափ տարածաժամանակի
և կոմպակտիֆիկացված լրացուցիչ չափանի տարածության կոմբինացիան պետք է
լինի բարձր չափանի Էյնշտեյնի դաշտի հավասարումների լուծում։ Մասնավոր
դեպքում, պետք է հնարավոր լինի վերականգնել հիմնական վիճակ հանդիսացող
լուծումը՝ բաղկացած Մինկովսկու 4-աչափ տարածությունից և d չափանի կոմպակտ
բազմաձևությունից։ Սա հստակ է Քլայնի տեսության դեպքում, երբ d=1։ Մեխանիզմը
ցավոք հստակ չէ ավելի բարձր չափանի տեսությունների դեպքում։

Ընդհանուր առմամբ, տարածաժամանակը հնարավոր է կոմպակտիֆիկացնել ցանկալի
ձևով, ձևափոխելով բարձր չափանի Էյնշտեյնի վակուումային լուծումները,
որն էլ հնարավոր կլինի մի քանի եղանակով․
\begin{enumerate}
\item ներառելով \textbf{ոլորում},
\item ավելացնելով բարձր աստիճանի ածանցյալներ Էյնշտեյնի հավասարումներին,
\item տեսությանն ավելացնելով բարձր չափանի էներգիա-իմպուլսի թենզոր։
\end{enumerate}
%
Վերոնշյալ կետերից 3-ի դեպքում, խելամիտ ընտրություն կատարելով հնարավոր
կլինի ստանալ լրացուցիչ չափերի «սպոնտան կոմպակտիֆիկացում»։ Ցավոք, այս
մոտեցումը զոհաբերում է Էյնշտեյնի և Կալուզայի երազանքը, որն էլ բնության՝
մաքուր երկրաչափական միացյալ տեսություն ունենալն է։

\section{D=11 ՍՈՒՊԵՐ-ԳՐԱՎԻՏԱՑԻԱ}

$\;$

Լրացուցիչ նյութական դաշտերը «ձեռքով» ավելացնելու ավելի բնական եղանակ
է տեսությունը սուպերսիմետրիկ դարձնելը, այսինքն, ամեն բոզոնի համապատասխանեցնել
դեռևս չբացահայտված ֆերմիոնային «սուպեր-զուգընկեր», և հակառակը։ Սրա
պատճառն այն է, որ Կալուզա-Քլայնի՝ բարձր չափանի տարածաժամանակային սիմետրիաները
որպես տրամաչափային սիմետրիա նկարագրելու ծրագիրը կարող է առաջացնել
միայն 4-աչափ տրամաչափային բոզոններ։ Եթե տեսությունում պահանջում ենք
ֆերմիոնային դաշտեր՝ սուպերսիմետրիայից ելնելով, ապա դրանց ավելացումը
պետք է լինի ձեռքով։ Վերջին նշված սահմանափակումը կարող է չվերաբերվել
ոչ-կոմպակտիֆիկացված Կալուզա-Քլայնի տեսություններին, որոնցում լրացուցիչ
կոորդինատներից «համեստ» կախվածությունը, որը փորձարարական սահմանափակման
արդյունք է, Էյնշտեյնի հավասարումներին տալիս է բավականին հարուստ կառուցվածք՝
բարձր չափանի մաքուր երկրաչափությունից ստանալ շատ ընդհանուր տիպի նյութ՝
4-աչափ տարածությունում։ Օրինակ, 5-աչափ տարածությունում էլեկտրամագնիսականության
տրամաչափային բոզոններից՝ ֆոտոններից բացի, ստացվում է փոշենման նյութ,
վակուում, կամ կոշտ նյութ։

Սուպերսիմետրիկ գրավիտացիան(կամ \textbf{սուպերգրավիտացիան}), կյանք
է առել որպես 4-աչափ տեսություն՝ 1976 թ․-ին, բայց կտրուկ ցատկ կատարեց
ավելի բարձր չափողականություններ, որը բավականին հաջողված ստացվեց D=11
դեպքում՝ 3 հիմնական պատճառով․
\begin{enumerate}
\item Նամը ցույց տվեց, որ 11-ը չափողականությունների առավելագույն թիվն է,
որը համատեղելի է միայնակ գրավիտոնի գաղափարի հետ։ Սրան հետագայում հաջորդեն
Ուիթթենի ապացույցը, որ 11-ը նաև նվազագույն թիվն է, որով հնարավոր է
Կալուզա-Քլայնի տեսությամբ միավորել տարրական մասնիկների ստարնդարտ մոդելի
բոլոր ուժերը(որպեսզի պարունակի ուժեղ փոխազդեցության $SU(3)$ և էլեկտրաթույլ
փոխազդեցության $SU(2)\times U(1)$ տրամաչափային խմբերը),
\item 11-ից ցածր չափողականությունների դեպքում լրացուցիչ նյութական դաշտերի
ընտրությունը միարժեք չէ, և Կրեմմերն ու այլոք 1978 թ․-ին ցույց տվեցին,
որ D=11 դեպքում ընտրություը միակն է, որը համատեղելի է սուպերսիմետրիայի
սզբունքին(կան հավասար Բոզե և Ֆերմի ազատության աստիճաններ),
\item Ֆրեյդը և Ռուբինը 1980 թ․-ին ցույց տվեցին, որ D=11 մոդելի կոմպակտիֆիկացումը
հնարավոր է միայն երկու եղանակով․ կոմպակտիֆիկացնելով 4, կամ 7 չափողականությունները
և մակրոսկոպական թողնել համապատասխանաբար մնացած 7, կամ 4 չափողականությունները։
\end{enumerate}
%
Ոգևորված այս հաջողություններից, 1980-ականներին 11 չափանի սուպեր-գրավիտացիայի
տեսությունը «Ամեն ինչի տեսության» առաջատար թեկնածուն էր։

Սակայն որոշ հանգամանքներ առաջ եկան, որոնք աղավաղում են այս տեսլականը։
Թվարկենք դրանցից մի քանիսը․
\begin{enumerate}
\item Կոմպակտ բազմաձևությունները, որոնք դիտարկված էին Ուիթթենի կողմից, պարզվեց
որ չեն առաջացնում քվարկեր և լեպտոններ և համատեղելի չեն սուպերսիմետրիայի
հետ։ Դրանց լավագույն փոխարինողերը 7 չափանի սֆերան և սեղմված 7 չափանի
սֆերան՝ նկարագրված համապատասխանաբար սիմետրիա $SO(8)$ և $SO(5)\times SU(2)$
խմբերով։ Ցավոք այս խմբերը չեն պարունակում ստանդարտ մոդելի սիմետրիայի
նվազագույն պայմանը՝ $SU(3)\times SU(2)\times U(1)$ խումբը։ Սա հիմնականում
ուղղվում է ավելացնելով նյութական դաշտեր, «բաղադրյալ տրամաչափային դաշտեր»՝
11 չափանի Լագրանժյանին։
\item Շատ բարդ է 11 չափանի տեսությունում կառուցել քիրալություն՝ ռեալիստիկ
ֆերմիոնային մոդելի համար։ Սրա համար արվել են փորձեր ավելացնել բարձր
չափանի տրամաչափային դաշտեր, ոչ-կոմպակտ ներքին բազմաձևություններ, և
Ռիմանյան երկրաչափություն։
\item D=11 սուպեր-գրավիտացիայի տեսությունը խաթարված է 4-աչափ տարածությունում
առկա կոսմոլոգիական հաստատունով, ոը հնարավոր չէ վերացնել։
\item Տեսության քվանտացումը բերում է անդարձելի անոմալիաների։
\end{enumerate}
%
Վերոնշյալ խնդիրներից որոշ մասը հնարավոր է վերացնել՝ տեսության չափողականությունն
իջեցնելով 10-ի․ հեշտանում է քիրալության կառուցումը և վերանում են անոմալիաներից
շատերը։ Սակայն, քիրալ ֆերմիոնների ներմուծումը հանգեցնում է անոմալիաների
\textbf{նոր տեսակների}։ D=11 տեսության միակությունը խափանվում է, և
բարձր էներգիաների տեսությունը բնականորեն չի բաժանվում 4 մակրոսկոպական
և 6 կոմպակտ չափողականությունների։ Եվ իսկապես, D=10 սուպերգրավիտացիայի
մոդելները ոչ միայն պահանջում են բարձր չափանի նյութական դաշտեր՝ կոմպակտիֆիկացիան
ապահովվելու համար, այլ նաև ամբողջովին անտեսում են Կալուզա-Քլայնի մեխանիզմով
առաջացած տրամաչափային դաշտերը, և ստիպված բոլոր տրամաչափային դաշտերը
տեսությանն են ավելացվում ձեռքով։ Եվ վերջապես, Կալուզայի՝ մաքուր երկրաչափական
նկարագրություն ունենալու սկզբունքը խախտվում է ամբողջությամբ։

\section{D=10 ՍՈՒՊԵՐ-ԼԱՐԵՐԻ ՏԵՍՈՒԹՅՈՒՆԸ}

$\;$

D=10 տեսությունում առկա անոմալիաների խնդրի լուծման մեջ առաջխաղացում
նկատվեց, երբ Գրինը և Շվարցը, ինչպես նաև Գրոսսն ու այլոք ցույց տվեցին,
որ կա 10 չափանի սուպեր-գրավիտացիայի ընդամենը 2 տեսություն, որոնցում
հնարավոր է կատարելապես վերացնել բոլոր անոմալիաները, որոնք համապատասխանաբար
հիմնված են $SO(32)$ և $E_{8}\times E_{8}$ խմբերի վրա։ Այս դեպքում
ևս պետք է ավելացվեն բարձր չափանի Լագրանժյանին, որոնք հայտնի են որպես
\textbf{Չապլին-Մանտոնի անդամներ}։ Սակայն այս անգամը այդպիսի անդամների
ավելացումն այդքան էլ կամայական ձևով չէ․ ավել անդամները նրանք էին,
որոնք այսպես թե այնպես առաջանալու էին ցածր էներգիաների դեպքում որոշ
տիպի սուպեր-լարերի տեսությամբ մոտարկումների դեպքում։

Կոմպակտիֆիկացված Կալուզա-Քլայնի տեսություններից սուպեր-գրավիտացիայի
տեսությունների փոխարեն նախընտրելի դառան սուպեր-լարերի տեսությունները։
Սուպեր-լարերը՝ լարերի սուպերսիմետրիկ ընդհանրացումները, խուսափում են
լարերի առաջին տեսությունների պատճառ հանդիսացած տախյոնների ընդհանուր
կանխատեսումներից, բայց պահպանում են դրանց լավագույն հատկությունը․
անոմալիաներից զուրկ գրավիտացիայի քվանտային տեսություն ունենալու հավանականությունը։
Վերջերս կապ է հաստատվել սուպեր-լարերի որոշ վիճակների և սև խոռոչների
միջև, և նույնիսկ առաջ է քաշվել տեսլական, որ սուպեր-լարերը կարող են
լուծել երկար ժամանակ բաց մնացած սև խոռոչների ինֆորմացիայի պարադոքսը։

Միակության խնդրի նման մի բանի առաջ է կանգնում D=10 սուպեր-լարերի տեսությունը,
քանի որ $SO(32)$ և $E_{8}\times E_{8}$ խմբերն առաջացնում են 5 տարբեր
լարերի տեսություններ։ Այս բարդությունը լուծվել է Ուիթթենի կողմից,
ով ցույց է տվել, որ այդ 5 տեսությունները հնարավոր է ներկայացնել որպես
1 տեսության կերպարանքներ, որը հիմա հայտնի է որպես Մ-տեսություն։ Այս
տեսության ցածր էներգիական սահմանը դառնում է D=11 սուպեր-գրավիտացիան։

\section{ՊՐՈՅԵԿՏՄԱՆ ՄՈՏԵՑՈՒՄԸ}

$\;$

Ավել չափողականությունների կոմպակտիֆիկացումը Կալուզայի գլանային պայմանի
միակ բացատրությունը չէ։ Այլ, քիչ հայտնի մոտեցում 1931 թ․-ին ներկայացրել
են Հոֆֆմանն ու Վեբլեն։ Նրանք ցույց տվեցին, որ 5-րդ չափողականությունը
«կլանվում է» հասարակ 4-աչափ տարածաժամանակում, եթե հարաբերականության
ընդհանուր տեսության դասական թենզորները փոխարինվեն պրոյեկտվածներով։
Նոր կոորդինատներ համարվելու փոխարեն , լրացուցիչ չափողականությունները
համարվեցին տեսանելի օգնություն։ Խնդրի այս լուծման «գինը» Էյնշտեյնի
տեսության երկրաչափական հիմքի փոփոխում է։ Այս միտքը գրավեց Ջորդանի,
Պաուլիի և այլ գիտնականների ուշադրությունը։ Տեսության սկզբնական տարբերակները
բախվել են փորձարարական սահմանափակումների՝ Բրանս-Դիկեի $\omega$ պարամետրի
հետ։

Պրոյեկտման մոտեցումը «վերակենդանացավ» 2 նոր ձևակերպումներում․
\begin{enumerate}
\item Լեսսները սկալյար դաշտին վերագրում է զուտ միկրոսկոպական իմաստ, որն
ունի հետաքրքիր հետևանքներ տարրական մասնիկների ֆիզիկայում,
\item Շմուցերն օժտել է վակուումը բարձր չափանի նյութով՝ «ոչ-երկրաչափայնացվող
սուբստրատով», որի արդյունքում զոհաբերվում է Էյնշտեյնի երազանքը՝ վերածելով
ֆիզիկան երկրաչափության։
\end{enumerate}
%
Վերոնշյալ 2-րդ կետն իրականացնում է փորձարկման ենթակա որոշ կանխորոշումներ,
որոնք դեռևս համատեղելի են դիտողական տվյալների հետ։

\section{ՈՉ-ԿՈՄՊԱԿՏԻՖԻԿԱՑՎԱԾ ՄՈՏԵՑՈՒՄԸ}

$\;$

Կոմպակտիֆիկացման և պրոկտման մոտեցումների այլընտրանքային տարբերակ է
ավել չափողականությունների՝ առանց կոմպակտիֆիկացման պայման դնելու հաշվի
առնելը, և ենթադրելը, որ բնությունը դրանցից միայն շատ քիչ է կախված,
ինչպես օրինակ Մինկովսկու դեպքում 4-րդ կոորդինատը՝ ոչ-ռելյատիվիստիկ
արագությունների դեպքում։ Այլ կերպ ասած այս դեպքում Կալուզայի գլանային
պայմանի հարցը լուծվում է այն «բաց թողնելով»։ Սակայն հարց է առաջանում,
թե ո՞րն է բնության՝ \textbf{համարյա գլանային} լինելու պատճառը։ Եթե
լրացուցիչ չափողականությունները տարածանման են, ապա հնարավոր է ենթադրելը,
որ մասնիկները, մեծ պոտենցյալային պատնեշով «բանտարկված են» 4-աչափ հիպերմակերևույթում։
Այս տիպի կարծիքներ շոշափվել են դեռևս 1962 թ․-ից։

Սահմանափակող պոտենցյալների գաղափարը կոմպակտիֆիկացման մեխանիզմի համեմատ
այդքան էլ ակնհայտ բարելավվում չէ։ Այլընտրանք է Մինկովսկու օրինակը
կրկնօրինակելը, այսինքն ավել չափողականությունների՝ օրինակ ժամանակի,
տարածանման չլինելը։ Այս դեպքում բնության՝ համարյա գլանային լինելը
պետք է փնտրել ավել կոորդինատների ֆիզիկական նկարագրությունում, օրինակ
այնպիսի արժեքներում, որոնք փոխում են չափողականությունը(ինչպես լույսի
c արագությունը՝ $ct\rightarrow x$) և տալիս տարածության միավորներ։
Այդպիսի առաջին առաջարկը 1983 թ․-ին արել է Վեսսոնը՝ «տարածություն-ժամանակ-զանգված»
տեսությամբ, ով առաջարկեց 5-րդ չափողականությունը կապել հանգստի զանգվածի
հետ՝ $x^{4}=\frac{Gm}{c^{2}}$ առնչությամբ։ 4-աչափ ֆիզիկայում այս
նոր կոորդինատի ազդեցությունը երևում է մասնիկների հանգստի զանգվածի
ժամանակից կախում ունենալուց։ Հետևյալ մոդելը խորն ուսումնասիրվել է
Վեսսոնի և ուրիշների կողմից՝ աստղաֆիզիկայում և կոսմոլոգիայում ունեցած
հետևանքների պատճառով, որն էլ այնուհետև Ֆուկուի կողմից տարածվել է 5-ից
ավել չափողականությունների վրա($\hbar$-ը և e-ն տանում են համապատասխանաբար
c-ի և G-ի դերերը)։

\section{ԿԱԼՈՒԶԱՅԻ ՄԵԽԱՆԻԶՄԸ}

$\;$

Կալուզան միավորեց էլեկտրամագնիսականությունը գրավիտացիային, կիրառելով
Էյնշտեյնի հարաբերականության ընդհանուր տեսությունը 5-աչափ տարածաժամանակային
բազմաձևության վրա՝ 4-աչափի փոխարեն։

5-աչափ տարածությունում Էյնշտեյնի հավասարումներն առանց էներգիա-իմպուլսի
5-աչափ թենզորի, ընդունում են 
\begin{equation}
\hat{G}_{AB}=0\label{eq:G1}
\end{equation}
տեսքը, կամ համարժեքորեն
\begin{equation}
\hat{R}_{AB}=0,\label{eq:R1}
\end{equation}
որտեղ $\hat{G}_{AB}\equiv\hat{R}_{AB}-\hat{R}\frac{\hat{g}_{AB}}{2}$
Էյնշտեյնի թենզորն է, $\hat{R}_{AB}$-ն 5-աչափ Ռիչիի թենզորն է, $\hat{R}=\hat{g}_{AB}\hat{R}^{AB}$
5-աչափ Ռիչիի սկալյարն է, իսկ $\hat{g}_{AB}$-ն 5-աչափ մետրիկական թենզորն
է։ Շարադրանքում լատինական մեծատառ A,B,... ինդեքսները փոփոխվում են
0-ից 4, իսկ գլխարկով անդամները 5-աչափ մեծություններ են։ Այս հավասարումները
ստացվում են վարիացիայի ենթարկելով Էյնշտեյնի 5-աչափ 
\begin{equation}
S=-\frac{1}{16\pi\hat{G}}\int\hat{R}\sqrt{-\hat{g}}d^{4}x\,dy\label{eq:Action}
\end{equation}
գործողությունը՝ ըստ 5-աչափ մետրիկայի, որտեղ $y=x^{4}$-ը նոր, 5-րդ
կոորդինատն է, իսկ $\hat{G}$-ն՝ 5-աչափ «գրավիտացիոն հաստատունը»։

Հետևյալ հավասարումներում նյութի բացակայությունը հենց Կալուզայի առաջին
ենթադրության մաթեմատիկական նկարգրությունն է՝ ոգեշնչված Էյնշտեյնից,
ըստ որի բարձր չափանի տիեզերքը դատարկ է։ Գաղափարը 4-աչափ նյութը 5-աչափ
տարածությունում որպես մաքուր երկրաչափության նկարագրումն է։

\section{ՀԱՐԱԲԵՐԱԿԱՆՈՒԹՅԱՆ ԸՆԴՀԱՆՈՒՐ ՏԵՍՈՒԹՅԱՆ ՆՎԱԶԱԳՈՒՅՆ ԸՆԴԼԱՅՆՈՒՄԸ}

$\;$

5-աչափ Ռիչիի թենզորն ու Քրիստոֆելի սիմվոլները սահմանված են ըստ մետրիկայի
այնպես, ինչպես 4-աչափ դեպքում․
\begin{align}
\hat{R}_{AB} & =\partial_{C}\hat{\Gamma}_{AB}^{C}-\partial_{B}\hat{\Gamma}_{AC}^{C}+\hat{\Gamma}_{AB}^{C}\hat{\Gamma}_{CD}^{D}-\hat{\Gamma}_{AD}^{C}\hat{\Gamma}_{BC}^{D},\nonumber \\
\hat{\Gamma}_{AB}^{C} & =\frac{1}{2}\hat{g}^{CD}\left(\partial_{A}\hat{g}_{DB}+\partial_{B}\hat{g}_{DA}-\partial_{D}\hat{g}_{AB}\right):\label{eq:RGamma}
\end{align}

Այժմ, ամեն ինչ կախված է 5-աչափ մետրիկայի ընտրությունից։ 5-աչափ մետրիկայում
$\alpha\beta$-ով մասը համապատասխանում է 4-աչափ մետրիկային, $\alpha4$-ով
մասը՝ էլեկտրամագնիսականության $A_{\alpha}$ պոտենցյալին, իսկ 44-ով
մասը՝ սկալյար $\phi$ դաշտին։ Պատկերենք 5-աչափ մետրիկայի տեսքը՝ հաշվի
առնելով վերջինս․
\begin{equation}
\left(\hat{g}_{AB}\right)=\left(\begin{array}{cc}
g_{\alpha\beta}+\kappa^{2}\phi^{2}A_{\alpha}A_{\beta} & \kappa\phi^{2}A_{\alpha}\\
\kappa\phi^{2}A_{\beta} & \phi^{2}
\end{array}\right),\label{eq:g1}
\end{equation}
որտեղ էլեկտրամագնիսականության $A_{\alpha}$ պոտենցյալը փոփոխված է
$\kappa$ հաստատունով, որպեսզի հետագայում գործողության մեջ ստացվեն
ճիշտ գործակիցներ։ Շարադրանքում հունական $\alpha,\beta,...$ փոքրատառ
ինդեքսները փոփոխվում են 0-ից 3, իսկ լատինական փոքրատառ a,b,... ինդեքսները՝
1-ից 3։ Ընտրված է միավորների նորմալ համակարգը, որում վերցված են $c=1,\hbar=1$
և դիտարկված է 4-աչափ մետրիկայի $(+---)$ տեսքը։

\section{ԳԼԱՆԱՅԻՆ ՊԱՅՄԱՆԸ}

$\;$

Եթե հաշվի առնենք Կալուզայի տեսության 3-րդ սկզբունքը և հաշվի չառնենք
ըստ 5-րդ կոորդինատի բոլոր ածանցյալները, ապա հաշվի առնելով \eqref{eq:g1}
մետրիկան և \eqref{eq:RGamma} նշանակումները, կստանանք, որ 5-աչափ դաշտի
\eqref{eq:R1} հավասարումների $\alpha\beta,\alpha4,$ և 44 անդամներն
ընդունում են
\begin{align}
G_{\alpha\beta} & =\frac{\kappa^{2}\phi^{2}}{2}T_{\alpha\beta}^{EM}-\frac{1}{\phi}\left[\nabla_{\alpha}\left(\partial_{\beta}\phi\right)-g_{\alpha\beta}\Box\phi\right],\nonumber \\
\nabla^{\alpha}F_{\alpha\beta} & =-3\frac{\partial^{\alpha}\phi}{\phi}F_{\alpha\beta},\nonumber \\
\Box\phi & =\frac{\kappa^{2}\phi^{3}}{4}F_{\alpha\beta}F^{\alpha\beta}\label{eq:Reduced}
\end{align}
տեսքերը, որոնցում $G_{\alpha\beta}\equiv R_{\alpha\beta}-\frac{Rg_{\alpha\beta}}{2}$
Էյնշտեյնի թենզորն է, $T_{\alpha\beta}^{EM}\equiv\frac{g_{\alpha\beta}F_{\gamma\delta}F^{\gamma\delta}}{4}-F_{\alpha}^{\gamma}F_{\beta\gamma}$
էլեկտրամագնիսական էներգիա-իմպուլսի թենզորն է, և $F_{\alpha\beta}\equiv\partial_{\alpha}A_{\beta}-\partial_{\beta}A_{\alpha}$։
Ընդհանուր առկա է 10+4+1=15 հավասարում, ինչն էլ որ սպասվում էր, քանի
որ 5-աչափ մետրիկայում կա 15 ազատ անդամ։

\section{$\phi=const$ ԴԵՊՔԸ}

$\;$

Եթե $\phi$ սկալյար դաշտը հաստատուն է ամբողջ տարածաժամանակում, ապա
\eqref{eq:Reduced}-ի առաջին 2 հավասարումները պարզապես Էյնշտեյնի և
Մաքսվելի հավասարումներն են․
\begin{align}
G_{\alpha\beta} & =8\pi G\phi^{2}T_{\alpha\beta}^{EM},\nonumber \\
\nabla^{\alpha}F_{\alpha\beta} & =0,\label{eq:EM}
\end{align}
որտեղ $\kappa$ պարամետրի արժեքն արտահայտվել է G գրավիտացիոն հաստատունի
միջոցով
\begin{equation}
\kappa\equiv4\sqrt{\pi G}\label{eq:KG}
\end{equation}
առնչությամբ։ Այս արդյունքը սացել են Կալուզան և Քլայնը, տեղադրելով
$\phi=1$ արժեքը։ Սակայն, $\phi=const$ պայմանը համատեղելեի է \eqref{eq:Reduced}
դաշտի հավասարումներից 3-ի հետ, երբ $F_{\alpha\beta}F^{\alpha\beta}=0$,
ինչը որ առաջին անգամ նշվել է Ջորդանի և Թիրիի կողմից։ Այն հանգամանքը,
որ սրա ճանաչման համար անհրաժեշտ եղավ 20 տարի, փաստում է խորը կասկածը՝
դեպի սկալյար դաշտերը։

Այժմ, նույն ստացումն արվում է վարիացիոն եղանակով։ Օգտագործելով \eqref{eq:g1}
մետրիկան, \eqref{eq:RGamma} սահմանումները, օգտագործելով գլանային
պայմանը ոչ միայն հաշվի չառնելու համար ըստ 5-րդ կոորդինատի ածանցյալներն,
այլ նաև գործողությունից դուրս հանելով ըստ դրա ինտեգրումը, \eqref{eq:Action}-ից
կստանանք
\begin{equation}
S=-\int d^{4}x\sqrt{-g}\phi\left(\frac{R}{16\pi G}+\frac{1}{4}\phi^{2}F_{\alpha\beta}F^{\alpha\beta}+\frac{2}{3\kappa^{2}}\frac{\partial^{\alpha}\phi\partial_{\alpha}\phi}{\phi^{2}}\right),\label{eq:MAction}
\end{equation}
որտեղ G-ն սահմանված է 5-աչափ գրավիտացիոն $\hat{G}$ հաստատունի միջոցով
\begin{equation}
G\equiv\frac{\hat{G}}{\int dy}\label{eq:G4}
\end{equation}
առնչությամբ, որտեղ մենք $16\pi G$ անդամն ինտեգրալատակ արտահայտության
մեջ մտցնելու համար օգտվել ենք \eqref{eq:KG} կապից։ Ինչպես նախկինում,
այնպես էլ այս դեպքում, եթե վերցնենք $\phi=const$, ապա \eqref{eq:MAction}-ի
առաջին 2 անդամները կլինեն Էյնշտեյն-Մաքսվելի գրավիտացիայի և էլեկտրամագնիսական
ճառագայթման գործողությունները, իսկ 3-րդ անդամը կլինի Քլայն-Գորդոնի
անզանգված սկալյար դաշտի գործողությունը։

Այն հանգամանքը, որ \eqref{eq:Action} գործողությունից ստացվում է \eqref{eq:MAction}
գործողությունը, կամ որ նույնն է՝ առնաց աղբյուրի դաշտի \eqref{eq:R1}
հավասարումներից ստացվում է \eqref{eq:Reduced}-ը՝ նյութական աղբյուրով,
փաստում է Կալուզա-Քլայնի տեսության հրաշքը։ Ցույց տրվեց, որ 4-աչափ
տարածության նյութը(այս դեպքում էլեկտրամագնիսական ճառագայթումը) ծնվում
է դատարկ 5-աչափ տարածաժամանակի մաքուր երկրաչափությունից։ Հետագա շարունակական
Կալուզա-Քլայնի տեսությունների նպատակն այս սկզբունքը նյութի այլ տեսակների
վրա ընդլայնելն է։

\section{$A_{\alpha}=0$ ԴԵՊՔԸ, ԲՐԱՆՍ-ԴԻԿԵԻ ՏԵՍՈՒԹՅՈՒՆԸ}

$\;$

Եթե չպահանջենք $\phi=const$ պայմանը, ապա Կալուզայի 5-աչափ տեսությունը
էլեկտրամագնիսական երևույթներից բացի կպարունակի նաև Բրանս-Դիկեի տիպի
սկալյար դաշտի տեսություն, որը պարզ է դառնում, երբ դիտարկում է էլեկտրամագնիսական
պոտենցյալների վերացումը՝ $A_{\alpha}=0$։ Առանց գլանային պայմանի,
սա ուղղակի կապված կլիներ կոորդինատների ընտրությունից, և չէինք կորցնի
ոչ մի հանրահաշվական ընդհանրություն։ $A_{\alpha}=0$ պայմանը ֆիզիկական
պայման է, որը սահմանափակում է դնում տեսության «գրավիտոն-սկալյար սեկտոր»-ի
վրա։

Սա ընդունելի է որոշ դեպքերում, օրինակ, համասեռ և իզոտրոպ իրավիճակում,
որի դեպքում ոչ-անկյունագծային անդամները «կնախընտրեին» ուղղություն,
կամ վաղ տիեզերքի մոդելներում, որտեղ դինամիկ կերպով գերակշռում է սկալյար
դաշտը։ \eqref{eq:g1}-ից անտեսելով $A_{\alpha}$ դաշտերը, մետրիկան
կընդունի 
\begin{equation}
\left(\hat{g}_{AB}\right)=\left(\begin{array}{cc}
g_{\alpha\beta} & 0\\
0 & \phi^{2}
\end{array}\right)\label{eq:gHI}
\end{equation}
տեսքը։ Հաշվի առնելով Կալուզայի՝ տեսության հիմքում դրած 3 պայմանները,
դաշտի \eqref{eq:R1} հավասարումները՝ \eqref{eq:Action} գործողությունը
պարզեցվում է 
\begin{equation}
S=-\frac{1}{16\pi G}\int d^{4}x\sqrt{-g}R\phi\label{eq:BDA0}
\end{equation}
տեսքի։ \eqref{eq:BDA0}-ը Բրանս-Դիկեի`
\begin{equation}
S_{BD}=-\int d^{4}x\sqrt{-g}\left(\frac{R\phi}{16\pi G}+\omega\frac{\partial^{\alpha}\phi\partial_{\alpha}\phi}{\phi}\right)+S_{m}\label{eq:BDA}
\end{equation}
գործողության մասնավոր դեպքն է՝ $\omega=0$ դեպքում։

$\omega$ պարամետրի արժեքը՝ հիմնվելով դիտողական տվյալների վրա, իհարկե,
մեծ է 500-ից, հետևաբար այս մոդելն այժմ հուսալի չէ։ Այս սահմանափակումը
հնարավոր է վերացնել, \eqref{eq:BDA} գործողությունում ավելացնելով
լրացուցիչ $V(\phi)$ ոչ 0-ական պոտենցյալային դաշտ, ինչպես, օրինակ
երկարացված ինֆլյացիոն և այլ տեսություններում, կամ, թույլատրել Բրանս-Դիկեի
$\omega$ պարամետրի՝ $\phi$-ից կախվածությունը, ինչպես օրինակ հիպերերկարացված
և այլ ինֆլյացիոն մոդելներում։

\section{ՔԼԱՅՆԻ ԿՈՄՊԱԿՏԻՖԻԿԱՑՄԱՆ ՄԵԽԱՆԻԶՄԸ}

$\;$

Կալուզայի գլանային պայմանը, որ ոչ մի ֆիզիկական մեծություն կախված չէ
5-րդ չափողականությունից, միացյալ տեսության հետևորդներին թվաց խորդ։
Քլայնն իր մոտեցումն առաջ քաշեց քվանտային տեսության ստեղծման ժամանակ,
և, ամենևին զարմանալի չէ, որ կապեց Կալուզայի գլանային պայմանի բացատրությունը
ավել չափողականության շատ փոքր լինելուն։

Քլայնը ենթադրեց, որ 5-րդ կոորդինատը տարածանման է, ինչպես առաջին 3
կոորդինատները, և վերագրեց 2 պայման․
\begin{enumerate}
\item շրջանային տոպոլոգիա($S^{1}$),
\item փոքր մասշտաբներ։
\end{enumerate}
%
Վերոնշյալ 1-ին պայմանի դեպքում կամայական $f(x,y)$ ֆունկցիա դառնում
է պարբերական, որտեղ $x=(x^{0},x^{1},x^{2},x^{3})$, և $y=x^{4}$՝
\[
f(x,y)=f(x,y+2\pi r),
\]
որտեղ r-ը 5-րդ չափողականության մասշտաբային պարամետր հանդիսացող «շառավիղն»
է։ Հետևաբար, բոլոր դաշտերը հնարավոր է ենթարկել Ֆուրիե վերլուծության․
\begin{align}
g_{\alpha\beta}(x,y) & =\sum_{n=-\infty}^{n=+\infty}g_{\alpha\beta}^{(n)}(x)e^{iny/r},\nonumber \\
A_{\alpha}(x,y) & =\sum_{n=-\infty}^{n=+\infty}A_{\alpha}^{(n)}(x)e^{iny/r},\nonumber \\
\phi(x,y) & =\sum_{n=-\infty}^{n=+\infty}\phi^{(n)}e^{iny/r},\label{eq:Fourier}
\end{align}
որտեղ ցուցիչում առկա (n)-ը հղում է անում n-րդ Ֆուրիե մոդին։ Քվանտային
տեսության շնորհիվ, այս մոդերն իրենց հետ y ուղղությամբ տանում են իմպուլս՝
$\frac{\mid n\mid}{r}$-ի կարգի։ Ահա այստեղ Քլայնի երկրորդ սկզբունքն
է ի հայտ գալիս․ եթե r-ը բավականին փոքր լինի, ապա նույնիսկ n=1 մոդի
y-իմպուլսն այնքան մեծ կլինի, որ փորձարարական սահմանից դուրս կգտնվի։
Հետևաբար, տեսանելի կլինեն միայն n=0 մոդերը, որոնք կախված չեն y-ից,
ինչը պահանջված էր Կալուզայի տեսությունում։

Հասկանանք, թե ինչքան մեծ կարող է լինել մասշտաբային r պարամետրը։ r-ի
արժեքի վրա ուժեղ սահմանափակումներ դրվում են բարձր էներգիաների տարրական
մասնիկների ֆիզիկայից, որը ստուգում է համեմատաբար մեծ զանգվածներ և
փոքր հեռավորություններ։ Այդպիսի մի քանի փորձեր r-ի վրա դնում են ատտոմետրից($10^{-18}$
մ) փոքր լինելու պայման։ Տեսաբանները հաճախ r-ին տալիս են Պլանկի երկարության
արժեք($l_{PL}\sim10^{-35}$ մ), որը թե՜ բնական ընտրություն է, և թե՜
այնքան փոքր է, որ n=0-ից տարբեր մոդերի զանգվածը լինի Պլանկի զանգվածից($m_{PL}\sim10^{19}$
ԳէՎ) շատ անգամներ մեծ։

Ընդհանուր դեպքում, Կալուզայի 5-աչափ \eqref{eq:g1} մետրիկան պարունակում
է բոլոր Ֆուրիե մոդերը։ Վերջինից, կոմպակտիֆիկացման տեսություններում
հնարավոր է ստանալ այսպես կոչված «Կալուզա-Քլայնի անսացը», որում դեն
են նետված $n\neq0$ մոդերը։ 5-աչափ դեպքում անսացը դեն է նետում $g_{\alpha\beta},A_{\alpha},\phi$
մեծությունների y-ից կախվածությունը, սկիզբ տալով $g_{\alpha\beta}^{(0)}$
գրավիտոնի, $A_{\alpha}^{(0)}$ ֆոտոնի և $\phi^{(0)}$ սկալյարի էֆեկտիվ,
4-աչափ, «ցածր էներգիական» տեսությանը։ Ավելի բարձր չափերի դեպքում սկզբնական
մետրիկայի և Կալուզա-Քլայնի անսացի միջև կապն այսքան պարզ չի ստացվում։

Կարևոր է նշել «հիմնական վիճակի մետրիկա» $\hat{g}_{AB}$-ի՝ $\hat{g}_{AB}(x,y)$-ի
վակուումային միջինը լինելու հանգամանքը, որը որոշում է կոմպակտ տարածության
տոպոլոգիան։ 5-աչափ դեպքում, որտեղ տոպոլոգիան $M^{4}\times S^{1}$
է, վերջին դիտարկումը կընդունի 
\begin{equation}
\left(\left\langle \hat{g}_{AB}\right\rangle \right)=\left(\begin{array}{cc}
\eta_{\alpha\beta} & 0\\
0 & -1
\end{array}\right)\label{eq:VEM}
\end{equation}
տեսքը, որտեղ $\eta_{\alpha\beta}$-ն Մինկովսկու 4-աչափ տարածության
մետրիկան է։
\begin{thebibliography}{1}
\bibitem{KKG} J.M. Overduin, P.S. Wesson, Kaluza-Klein gravity, Physics
Reports Volume 283, Issues 5–6, April 1997, Pages 303-378

\end{thebibliography}

\end{document}
