%% LyX 2.3.7 created this file.  For more info, see http://www.lyx.org/.
%% Do not edit unless you really know what you are doing.
\documentclass[12pt,a4paper]{article}
\usepackage{amsmath}
\usepackage{amsthm}
\usepackage{fontspec}
\setmainfont[Mapping=tex-text]{Times New Roman}
\setsansfont[Mapping=tex-text]{Times New Roman}
\setmonofont{Times New Roman}

\makeatletter
\@ifundefined{date}{}{\date{}}
%%%%%%%%%%%%%%%%%%%%%%%%%%%%%% User specified LaTeX commands.
\usepackage{geometry}
 \geometry{
 right=12mm,
 left=12mm,
 top=17.5mm,
 bottom=22.5mm
 }
\geometry{verbose}
\renewcommand{\baselinestretch}{1.5} 
\sloppy

\makeatother

\usepackage{polyglossia}
\setdefaultlanguage[variant=american]{english}
\begin{document}
\title{ԿԱԼՈՒԶԱ-ՔԼԱՅՆԻ ԳՐԱՎԻՏԱՑԻԱ}
\author{Չալյան Գոռ Վարդանի\\
 Տեսական Ֆիզիկայի ամբիոն\\
 Մագիստրատուրա 1 կուրս}

\maketitle
 

\part{Ներածություն}

$\;$

Կալուզայի ձեռքբերումը 5-աչափ տարածությունում ձևակերպված Հարաբարականության
ընդհանուր տեսության՝ Էյնշտեյնի 4-աչափ գրավիտացիայի, ինչպես նաև Մաքսվելի
էլեկտրամագնիսականության տեսություններն իր մեջ պարունակել ցույց տալն
էր։ Սակայն նա մտցրեց արհեստական թվացող մի պայման, որը կոչվում է այսպես
կոչված \textbf{գլանային պայման}՝ դրված նոր մտցված 5-րդ կոորդինատի
վրա։ Վերջինս նշանակում է, որ ուղղակի ազդեցություն չունի ֆիզիկայի օրենքների
վրա։ Քլայնի ներդրումն այս պայմանի քիչ արհեստական դարձնելն էր, առաջարկելով
կոմպակտիֆիկացնել նոր՝ 5-րդ չափողականությունը։ Այս գաղափարը մեծ ցնծությամբ
ընդունվեց միացյալ տեսություն ունենալ ցանկացողների շրջանում, և երբ
հերթն արդեն հասել էր ուժեղ և թույլ փոխազդեցություններն էլ միավորելուն,
ենթադրվեց, որ ըստ Կալուզայի տրամաբանության ավելացված ավել չափողականությունները
ևս պետք է լինեն կոմպակտ։ Այս մտածելակերպը հիմք դարձավ 1980-ականներին
11 չափանի սուպեր-գրավիտացիայի տեսության, իսկ հետագայում՝ «Ամեն ինչի
տեսության»։

Կալուզան միավորեց ոչ միայն էլեկտրամագնիսականությունն ու գրավիտացիան,
այլ նաև նյութը և տարածության երկրաչափությունը՝ 4-աչափ տարածությունում
գտնվող ֆոտոնը դիտարկելով որպես դատարկություն՝ 5-աչափ տարածությունում։
Ժամանակակից Կալուզա-Քլայնի տեսությունները պահանջում են բարձր չափողականությամբ
նյութի դաշտերի առկայություն՝ հաջող կոմպակտիֆիկացիա համար։ 

Հարց է առաջանում, արդյո՞ք սա պարտադիր պայման է, թե՞ ոչ։ Պատասխանն
«այո» է, եթե դնում ենք ավել չափողականությունների \textbf{իրական},
\textbf{կոմպակտ} և \textbf{տարածանման} լինելու պայմանները։ Սակայն
գոյություն ունեն այլ տեսություններ, որոնցում այս պայմանները չեն պահանջվում,
օրինակ \textbf{պռոյեկտվող տեսությունները,} որոնցում չի պահանջվում
ավել չափողականությունների իրական լինելը, կամ \textbf{ոչ-կոմպակտիֆիկացված
տեսությունները}, որոնցում չի պահանջվում ավել չափողականությունների
տարածանման, կամ կոմպակտ լինելը։
\begin{thebibliography}{1}
\bibitem{key-1}

\end{thebibliography}

\end{document}
